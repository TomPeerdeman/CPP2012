	\section{Efficient Event-based Resource Discovery}
		This specific algorithm makes use of publish/subscribe messages, which need three key elements.
		Subscribers, or resource discovery clients, who want specific information or resources. Publishers, or 
		resource providers, who provide specific information or resources. And brokers, or broker networks.
		Brokers are used for matching subscribers and publishers and serves as a router between them.\\
		\\
		What makes this resource discovery interesting is that it considers if the resource is static or 
		dynamic. Besides this consideration it also is necessary to know if resources are used only once or 
		that they are going to be used continuously. When resources are needed continuously the algorithm 
		checks if in the time the subscriber is still busy, new resources are found that the subscriber might 
		need. \\
		\\
		The system makes use of a so called 'Modeltype'. For resources this Modeltype can have the 
		value 'Static' or 'Dynamic'. When a Subscriber needs resources he can ask for a Modeltype. The four 
		possible Modeltypes he can ask for are: 'Static', 'Dynamic', 'Static Continuous' and 'Dynamic 
		Continuous'. Each publisher tells what resource he has to offer. Static resources are denoted with 
		numerical operators ($<$, $\le$, $>$, $\ge$, $=$), whereas dynamic resource don't have a numerical operator. For 
		example, a static storage disk with 200GB could look like this: [disk,$\le$,200]. For a dynamic storage 
		disk this would look like [disk, 200].
		
		\subsection{Resource discovery}
		There is a difference in resource discovery when the resource is static and when the resource is 
		dynamic. When the resource added to the network is static, the publisher of this resource floods his 
		resource description (Modeltype, OS, storage memory etc.) to all the brokers in the network in an 
		advertisement. The brokers then cache this resource information. When a subscriber wants a specific 
		resource, all it has to do is send a message to the broker and the broker will reply with the resource 
		information he got earlier.\\
		\\
		When the resource is dynamic, the publisher will only send his resource 
		description to a single broker. The broker will recognize this, because the Modeltype will be 'Dynamic'. When 
		the resource description is updated, it only has to be changed in the broker. When a dynamic 
		resource is needed, a message is sent to brokers that match the resources. These brokers are 
		referred to as 'edge brokers'. These edge brokers return the latest resource description to the 
		subscriber, as long as the description still matches the conditions set by the subscriber.
		
		\subsection{Continuous usage}
		When the subscribers wants a continuous use of resources, there are slight differences. When the 
		subscriber wants to continuously use static resources, the broker remembers that the subscriber 
		issued a 'Static Continuous' Modeltype request. The resource given to the subscriber will be put in a 
		table, and will continuously be given to the subscriber. The subscriber therefore does not have to 
		issue new requests. \\
		\\
		When a subscriber wants to use dynamic resources continuously, the algorithm 
		works practically the same as with the non-continuous algorithm. The only difference is that when 
		the resource changes, this is updated and sent back to the subscriber.
		When a subscriber does not want to use the resource anymore, he can issue a message telling the 
		broker that he is done using that resource.
