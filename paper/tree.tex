\section{Tree bitmap}
  This algorithm proposes to use a hierarchic way for resource discovery,
  using a tree structure with bitmaps to represent resource information.
  All the information and attributes of a resource are transformed into
  bitmap representations. The algorithm makes a difference between
  Quantitative attributes and Qualitative attributes. Quantitative
  represent attributes like memory size or CPU speed. For all these
  attributes a best fit must be found, so that all users do not interfere
  with each other�s requests. The bitmap value of the Quantitative attributes
  is set to the maximumvalue of the range + 2 bits. If the value of one of the
  bits in the bitmap is zero, then that resource does not have that attribute.
  The Qualitative attributes represents information like the Operating System.
  The bitmap value for the Qualitative attribute is set to the size of the
  attribute set. Again, a one represents that the resource has this attribute,
  a zero represents that the resource does not have the attribute.
