\section{Tree bitmap}
  This algorithm proposes to use a hierarchic way for resource discovery,
  using a tree structure with bitmaps to represent resource information.
  All the information and attributes of a resource are transformed into
  bitmap representations. The algorithm makes a difference between
  Quantitative attributes and Qualitative attributes. Quantitative
  represent attributes like memory size or CPU speed. For all these
  attributes a best fit must be found, so that all users do not interfere
  with each other�s requests. The bitmap value of the Quantitative attributes
  is set to the maximumvalue of the range + 2 bits. If the value of one of the
  bits in the bitmap is zero, then that resource does not have that attribute.
  The Qualitative attributes represents information like the Operating System.
  The bitmap value for the Qualitative attribute is set to the size of the
  attribute set. Again, a one represents that the resource has this attribute,
  a zero represents that the resource does not have the attribute.

\subsection{The tree}
  As mentioned, the algorithm makes use of a tree structure.
  �Each node inside the tree represents a grid site�\cite{chang2010resource}.
  To make communication simple, each node only has to remember
  the IP addresses of their parent and, if they have any, their children.
  All the nodes that are not leaves are called �Index Servers� or IS-nodes.
  Every IS-node must know the resource information of their children, using the
  bitmaps provided by them. Every resource also has a status, free or occupied.
  When a request is made for a resource, the query is sent to a tree node.
  This can be either a IS-node or a leaf node.  The node looks if it matches the request.
  If not the message is forwarded to the parent node. If an IS matches a request
  to resource information of one of his children nodes, the request is sent down
  the tree towards the child. When finally the resource is found, the resource
  will be reserved. The status then changes to occupied, until the user releases
  the resource. This is updated in the parent IS-node.
  
\subsection{Data structure and Updating} 
  Each IS-node has two bitmaps, one local resource and one index bitmap.
  It also has a counter array, to store the resource information in.
  Leaf nodes only have a local resource bitmap. The index bitmap stores the
  resource information of its children. The counter array represents the
  amount of each resource the IS-node knows it has in its children. When a
  resource is occupied, the counter array will update, so that the IS-node
  knows it cannot use that resource anymore. This is done by lowering
  the correct counter value with one. When a resource is reclaimed, it
  has to update all the changed values. When there was a change in disk
  space for example. The node therefore sends the new information up to
  its parent node. When any update happens, the parent node is locked,
  to prevent race conditions. When the node is unlocked, the update is
  completed. Some attributes change more frequently. Therefore these
  attributes are given a threshold. If this threshold is not passed,
  no update will occur.
  
\subsection{Failed Nodes}
  Failed nodes are detected by two measurements. Firstly, if the user
  request is not acknowledged in a specific time period, the node that
  did not received the acknowledgement considers the node as down.
  Secondly, parent nodes and children nodes send each other messages
  once in a while, to inform that everything is ok. To make sure that
  no information is lost if a IS-node is fails, a backup is saved in
  one of its children. If the backup node fails, a new backup is made.
  If the IS-node fails, the child will inform all the former children
  of its failure, and will take its place as parent node.

