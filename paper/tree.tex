\section{Tree bitmap}
  This algorithm proposes to use a hierarchic way for resource discovery,
  using a tree structure with bitmaps to represent resource information.
  All the information and attributes of a resource are transformed into
  bitmap representations. The algorithm makes a difference between
  Quantitative attributes and Qualitative attributes. Quantitative
  represent attributes like memory size or CPU speed. For all these
  attributes a best fit must be found, so that all users do not interfere
  with each other�s requests. The bitmap value of the Quantitative attributes
  is set to the maximumvalue of the range + 2 bits. If the value of one of the
  bits in the bitmap is zero, then that resource does not have that attribute.
  The Qualitative attributes represents information like the Operating System.
  The bitmap value for the Qualitative attribute is set to the size of the
  attribute set. Again, a one represents that the resource has this attribute,
  a zero represents that the resource does not have the attribute.

\subsection{The tree}
  As mentioned, the algorithm makes use of a tree structure.
  �Each node inside the tree represents a grid site�\cite{}.
  To make communication simple, each node only has to remember
  the IP addresses of their parent and, if they have any, their children.
  All the nodes that are not leaves are called �Index Servers� or IS.
  very IS must know the resource information of their children, using the
  bitmaps provided by them. Every resource also has a status, free or occupied.
  When a request is made for a resource, the query is sent to a tree node.
  This can be either a IS or a leaf node.  The node looks if it matches the request.
  If not the message is forwarded to the parent node. If an IS matches a request
  to resource information of one of his children nodes, the request is sent down
  the tree towards the child. When finally the resource is found, the resource
  will be reserved. The status then changes to occupied, until the user releases
  the resource. This is updated in the parent IS.
