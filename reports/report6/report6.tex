\documentclass[a4paper]{article}
\usepackage{caption}
\usepackage{fancyhdr}
\usepackage[usenames, dvipsnames]{xcolor}
\usepackage{graphicx,hyperref,amsmath,float,subfigure,soul}
\usepackage[top=3cm,bottom=3cm,left=3cm,right=3cm]{geometry}
\hypersetup{
	colorlinks,
	citecolor=black,
	filecolor=black,
	linkcolor=black,
	urlcolor=black
}
\newcommand{\HRule}{\rule{\linewidth}{0.5mm}}
\pagestyle{fancy}
\lfoot{\small \color{gray}Tom Peerdeman - 10266186}
\cfoot{\thepage}
\rfoot{\small \color{gray}Ren\'e Aparicio Sa\'ez - 10214054}
\lhead{\small \color{gray} CUDA}
\begin{document}
	\begin{titlepage}
	\begin{center}
		\textsc{\Large Concurrency \& Parallel Programming}\\[0.5cm]
		\HRule \\[0,4cm]
		\textsc{\huge \bfseries CUDA}
		\HRule \\[8cm]
		\begin{minipage}{0.4\textwidth}
			\begin{flushleft}\large
				\emph{Auteurs: Tom Peerdeman \& Ren\'e Aparicio Saez}\\
			\end{flushleft}
		\end{minipage}
		\begin{minipage}{0.4\textwidth}
			\begin{flushright}\large
			\emph{Datum: 08-12-2012\\\hspace{1cm}}\\
			\end{flushright}
		\end{minipage}
	\end{center}
	\end{titlepage}

\section{Assignment 6.1 - Wave simulation}
  
	\subsection{Results}
		
		
		\begin{table}[H]
			\label{table:i_change_1000}
			\caption{Calculation times of the CUDA driven wave simulation in ms using a t\_max of 1000.}
			\begin{center}
				\begin{tabular}{| c | c | c | c | c |}
					\hline
					\multicolumn{5}{|l|}{t\_max = 1000, 512 threads per block}\\
					\hline
					i=1e3 & i=1e4 & i=1e5 & i=1e6 & i=1e7\\ 
					\hline
					4,95 & 5,05 & 4,94 & 3,36 & 3,32\\ 
					\hline
					3,41 & 3,57 & 3,32 & 3,35 & 3,31\\ 
					\hline
					3,42 & 3,54 & 3,34 & 3,36 & 3,3\\ 
					\hline
					4,91 & 5,05 & 3,24 & 3,35 & 3,31\\ 
					\hline
					3,49 & 3,47 & 3,3 & 3,37 & 3,3\\ 
					\hline
					4,91 & 3,49 & 3,34 & 3,36 & 3,31\\ 
					\hline
					4,91 & 3,5 & 4,89 & 3,41 & 3,32\\ 
					\hline
					4,93 & 5,02 & 4,94 & 3,37 & 3,3\\ 
					\hline
					3,46 & 5,06 & 4,95 & 3,38 & 3,31\\ 
					\hline
					4,93 & 5,01 & 4,92 & 3,34 & 3,3\\ 
					\hline
					\multicolumn{5}{|l|}{Average over 10 runs:}\\
					\hline
					4,332 & 4,276 & 4,118 & 3,365 & 3,308\\ 
					\hline
				\end{tabular}
			\end{center}
		\end{table}
		
		\begin{table}[H]
			\label{table:i_change_100}
			\caption{Calculation times of the CUDA driven wave simulation in $\mu$s using a t\_max of 100.}
			\begin{center}
				\begin{tabular}{| c | c | c | c | c |}
					\hline
					\multicolumn{5}{|l|}{t\_max = 100, 512 threads per block}\\
					\hline
					i=1e3 & i=1e4 & i=1e5 & i=1e6 & i=1e7\\ 
					\hline
					529 & 555 & 374 & 392 & 399\\ 
					\hline
					391 & 383 & 373 & 397 & 406\\ 
					\hline
					534 & 554 & 367 & 405 & 396\\ 
					\hline
					530 & 397 & 541 & 391 & 574\\ 
					\hline
					539 & 540 & 542 & 403 & 395\\ 
					\hline
					388 & 395 & 373 & 396 & 383\\ 
					\hline
					538 & 384 & 542 & 404 & 383\\ 
					\hline
					379 & 387 & 538 & 396 & 399\\ 
					\hline
					382 & 554 & 379 & 395 & 392\\ 
					\hline
					527 & 545 & 375 & 590 & 386\\ 
					\hline
					\multicolumn{5}{|l|}{Average over 10 runs:}\\
					\hline
					473,7 & 469,4 & 440,4 & 416,9 & 411,3\\ 
					\hline
				\end{tabular}
			\end{center}
		\end{table}
		
	\subsection{Speed comparison}
		If we compare our CUDA implementation to the previous implementations based on Threads, MPI and openMP, we can see in table \ref{table:speedComparison} that out CUDA implementation is by far the fastest.
		This result isn't suprising, the GTX 480 used has 15 SM units.
		Each of this SM units contain 32 ALU's, so in theory we can run 15 * 32 = 480 threads at the same time.
		The next fastest is the MPI implementation with 8 nodes running each 8 processes, this counts up as 64 processes/threads.
		Lets say this implementation scales up perfectly by time, so if we use 7.5 times more threads/processes the time would be 7.5 times lower.
		The implementation would use 480 threads/processes then and the time would be 16.0187 ms.
		This time would still be almost 5 times slower than the CUDA implementation, while using the same amount of threads.\\
		\\
		The sequential implementation in table \ref{table:speedComparison} is given by the pThreads implementation using 1 thread.
		This method is truly sequential since it is executed by 1 thread, but the timing also counts some overhead given by the pThreads.
		Given this we should keep in mind that a pure sequential implementation would be slightly faster than the given time under sequential in table \ref{table:speedComparison}.
		Keeping this in mind, we see that the CUDA implementation is over a thousand times faster than the (almost) sequential implementation.
		
		\begin{table}[H]
			\label{table:speedComparison}
			\caption{Speed comparison between pthreads, MPI, openMP and CUDA, t\_max = 1000, i\_max = 1e6.}
			\begin{center}
				\begin{tabular}{| l | c | c |}
					\hline
					Method & Average time (ms) & CUDA speedup\\
					\hline
					CUDA - 512 threads per block & 3.365 & 1.0\\
					\hline
					MPI - 8 nodes with 8 processes each & 120.140 & 35.703\\
					\hline
					MPI - 8 nodes with 1 process each & 495.634 & 147.291\\
					\hline
					OpenMP - 8 threads - static scheduler & 661.320 & 196.529\\
					\hline
					pThreads - 8 threads & 677.751 & 201.412\\
					\hline
					MPI - 1 node 8 processes & 1186.726 & 352.667\\
					\hline
					pThreads - 1 thread / Sequential & 3788.914 & 1125.977\\
					\hline
				\end{tabular}
			\end{center}
		\end{table}
	
	
	
	\subsection{Block sizes}
	
	\subsection{Results comparison}
	
	

\section{Assignment 6.2 - Parallel reduction}

\end{document}
