\documentclass[a4paper]{article}
\usepackage{fancyhdr}
\usepackage[usenames, dvipsnames]{xcolor}
\usepackage{graphicx,hyperref,amsmath,float,subfigure,soul}
\usepackage[top=3cm,bottom=3cm,left=3cm,right=3cm]{geometry}
\hypersetup{
	colorlinks,
	citecolor=black,
	filecolor=black,
	linkcolor=black,
	urlcolor=black
}
\newcommand{\HRule}{\rule{\linewidth}{0.5mm}}
\pagestyle{fancy}
\lfoot{\small \color{gray}Tom Peerdeman - 10266186}
\cfoot{\thepage}
\rfoot{\small \color{gray}Ren\'e Aparicio Sa\'ez - 10214054}
\lhead{\small \color{gray} Multithreaded Programs}
\begin{document}
	\begin{titlepage}
	\begin{center}
		\textsc{\Large Concurrency \& Parallel Programming}\\[0.5cm]
		\HRule \\[0,4cm]
		\textsc{\huge \bfseries Multithreaded Programs}
		\HRule \\[8cm]
		\begin{minipage}{0.4\textwidth}
			\begin{flushleft}\large
				\emph{Auteurs: Tom Peerdeman \& Ren\'e Aparicio Saez}\\
			\end{flushleft}
		\end{minipage}
		\begin{minipage}{0.4\textwidth}
			\begin{flushright}\large
			\emph{Datum: 09-11-2012\\\hspace{1cm}}\\
			\end{flushright}
		\end{minipage}
	\end{center}
	\end{titlepage}

  \section{Assignment 1.1 - Wave simulation}
  \subsection{Table with results}
    Tests on DAS4 are run for i = 1.000.000 and t = 1.000.
    The amount of threads used to generate the waves is increased to measure the
    difference in speed for the program.
    Each amount of threads is run 12 times. 
    The highest value and the lowest value are disregarded. 
    The remaining data is used to plot a graph.\\\\
    \begin{tabular}{| c | c | c | c | c | c | c | c |}
      \hline
      \multicolumn{4}{|c}{i = 1,000,000} & \multicolumn{4}{c|}{t = 1,000}\\
      \hline
      1 thread & 2 threads & 3 threads & 4 threads & 5 threads & 6 threads & 7 threads & 8 threads\\
      \hline
      \st{3.61657} & \st{1.68174} & 1.67706 & 0.967125 & \st{0.955026} & 0.707557 & 0.907312 & 0.681091\\
      \hline
      3.68809 & 1.69923 & 1.62901 & 0.952905 & 1.11722 & 0.783331 & 0.888922 & 0.677461\\
      \hline
      3.68735 & 1.68316 & 1.71819 & 0.946174 & 1.10978 & \st{0.802315} & \st{0.919069} & 0.652223\\
      \hline
      3.75564 & 1.71218 & 1.66693 & \st{0.92198} & 1.19389 & 0.722481 & 0.87717 & 0.736193\\
      \hline
      3.82117 & 1.70358 & 1.69521 & 0.951281 & 1.18303 & 0.791841 & 0.895334 & 0.656148\\
      \hline
      3.85017 & 1.69776 & 1.74229 & \st{1.28526} & 1.18654 & 0.69405 & 0.900644 & 0.661786\\
      \hline
      3.74723 & 1.70488 & 1.73762 & 0.972624 & \st{1.21719} & 0.794964 & 0.91415 & 0.666394\\
      \hline
      3.80248 & \st{1.71702} & 1.6256 & 0.9513 & 1.18918 & 0.746098 & 0.888725 & \st{0.951125}\\
      \hline
      3.79577 & 1.69529 & \st{1.85265} & 0.949003 & 1.16967 & 0.789424 & 0.894571 & 0.66652\\
      \hline
      3.86725 & 1.70905 & \st{1.45095} & 0.958044 & 1.17816 & 0.716474 & 0.899676 & 0.710156\\
      \hline
      \st{3.91954} & 1.70727 & 1.81612 & 0.964905 & 1.19274 & 0.733274 & \st{0.882675} & \st{0.640491}\\
      \hline
      3.87399 & 1.70378 & 1.82773 & 0.960786 & 1.20634 & \st{0.682611} & 0.8860616 & 0.669534\\
      \hline
      \multicolumn{8}{|l|}{Average of the remaining 10:}\\
      \hline
      3.788914 & 1.701978 & 1.713576 & 0.9564147 & 1.173655 & 0.7479494 & 0.89525656 & 0.6777506\\
      \hline
    \end{tabular}
    \begin{center}
      \includegraphics[width=0.9\textwidth]{speedplot.png}
    \end{center}
    Apparantly the performance is better if an even number of threads is used.
  \section{Assignment 1.2 - Sieve of Erasthotenes}
    On DAS4 (and on the uva computers) an error occurs.
    There are too many threads alive to create a new thread. This is understandable, because
    a new thread is created for every prime number found by the program.
    Which results in quite a high amount of threads.
    The program freezes and had to be terminated manually.
    
\end{document}
